\documentclass[a4paper,11pt]{article}

%%A Few Useful Packages
%\usepackage{marvosym}
%\usepackage{fontspec} 					%for loading fonts
%\usepackage{xunicode,xltxtra,url,parskip} 	%other packages for formatting
\usepackage{charter}
\usepackage{color,graphicx}
%\usepackage[usenames,dvipsnames]{xcolor}
%\usepackage[big]{layaureo} 				%better formatting of the A4 page
% an alternative to Layaureo can be ** \usepackage{fullpage} **
\usepackage{supertabular} 				%for Grades
\usepackage{titlesec}					%custom \section

%Setup hyperref package, and colours for links
\usepackage{hyperref}
\definecolor{linkcolour}{rgb}{0,0.2,0.6}
\hypersetup{colorlinks,breaklinks,urlcolor=linkcolour, linkcolor=linkcolour,pdfstartview=FitH}



%
%\setlength{\oddsidemargin}{1cm}
%\setlength{\marginparwidth}{1cm}
%% 
%% calculate other dimensions [textwidth and evensidemargin] 
%% in function of oddsidemargin and marginparwidth: 
%% would be nicer to put in the class file...
%%
%\addtolength{\marginparwidth}{-\marginparsep}
% \setlength{\evensidemargin}{\oddsidemargin}
% \setlength{\textwidth}{\paperwidth}
% \addtolength{\textwidth}{-2in}
% \addtolength{\textwidth}{-2\oddsidemargin}
% \addtolength{\textwidth}{\marginparwidth}
% \addtolength{\textwidth}{\marginparsep}
%

 \textheight 250mm  \textwidth 165mm
 \oddsidemargin=0pt \voffset=-1.7cm \hoffset=0.5cm

\setlength{\topmargin}{0.2in}


%CV Sections inspired by: 
%http://stefano.italians.nl/archives/26
\titleformat{\section}{\Large\scshape\raggedright}{}{0em}{}[\titlerule]
\titlespacing{\section}{0pt}{3pt}{3pt}
%Tweak a bit the top margin
%\addtolength{\voffset}{-1.3cm}
%Italian hyphenation for the word: ''corporations''
\hyphenation{im-pre-se}


%--------------------BEGIN DOCUMENT----------------------
\begin{document}

\pagestyle{empty} % non-numbered pages

%--------------------TITLE-------------
\par{\centering
		{\Huge Iryna  {Chaikovska}
	}\bigskip\par}

%--------------------SECTIONS-----------------------------------
%Section: Personal Data
\section{Personal Data}

\begin{tabular}{rl}
   \textsc{Place and Date of Birth:} & Kozyatyn, Ukraine  | 11 April 1985 \\
 
    \textsc{Address:}   & 13, rue des Berg\`eres,  91940 Les Ulis\\ & 
France \\
    \textsc{Cell Phone:}     & +33 (0)61 129 65 25 \\
    \textsc{Phone:}     & +33 (0)1 64 46 83 28\\
    \textsc{email:}   & \href{mailto:chaikovs@lal.in2p3.fr}{chaikovs@lal.in2p3.fr}\\
    % & \href{mailto:chaikovska@gmail.com}{chaikovska@gmail.com}
    								  
\end{tabular}

%Section: Work Experience at the top
\section{Work Experience}
\begin{tabular}{r|p{12cm}}
 \textsc{Jan 2013-- } & Laboratoire de l'Acc\'el\'erateur Lin\'eaire, IN2P3/CNRS\ \\ 
 \textsc{Dec 2014}&\emph{ Researcher (fixed-term contract) }\\ 
 &\footnotesize{Developing the high level applications integrated in the global control system for the ThomX machine (accelerator based compact Compton X-ray source to be built in Orsay). Working on the diagnostics for the PHIL (PHotoinIector at LAL) and ThomX facilities. Participating in the experimental activity aiming to demonstrate the intense flux of the gamma rays produced by the Compton scattering. Studying the beam dynamics under the Compton scattering.  } \\  
  \textsc{Dec 2014--} & Laboratoire de l'Acc\'el\'erateur Lin\'eaire, IN2P3/CNRS\ \\ 
 \textsc{present}&\emph{ Research Engineer (one year trial period) }\\ 
 &\footnotesize{Developing the high level applications integrated in the global control system for the ThomX machine (accelerator based compact Compton X-ray source to be built in Orsay). Working on the diagnostics for the PHIL (PHotoinIector at LAL) and ThomX facilities. Studying the beam dynamics under the Compton scattering. Working on a reliable positron source for ILC and CLIC with the hybrid target. } \\  
 \multicolumn{2}{c}{} \\
 
\end{tabular}

%Section: Education
\section{Education}
\begin{tabular}{rl}	
 \textsc 2009--2012 & Ph.D. in Physics, University Paris-Sud XI, Laboratoire de l'Acc\'el\'erateur Lin\'eaire\\
& \emph{Thesis: ``Polarized positron sources for the future Linear Colliders'' } \\
&\textsc{Diploma with honors.} {Scientific adviser: Dr. Alessandro Variola}\\
&\\
 \textsc 2008--2009 & M.Sc., in Fundamental and Applied Physics, University Paris-Sud XI \\&{Master 2,}  {Noyaux, Particules, Astroparticules et Cosmologie}\\ 
& \emph{Stage de pre-th\'ese: ``Optimization of the Compton line for the polarized positron sources'' } \\&  {Scientific adviser: Dr. Alessandro Variola}\\
&\\	
2006--2008 & M.Sc., in Solid State Physics,  National Taras Shevchenko University of Kyiv, Ukraine\\ 
& \emph{Thesis: ``Electronic and magnetic structure and properties of  Me-Al-C} \\&{ alloys (Me = Fe, Mn, Co, Ni)'' } \\
&  \textsc{Diploma with honors.} {Scientific adviser: Dr. Sc.  Vladislav Andrushchenko}\\
&\\
2002--2006 & B.Sc., in Solid State Physics,  National Taras Shevchenko University of Kyiv, Ukraine\\ 
& \emph{Thesis: `` Electro-optics of nematic liquid crystals doped with  ferroelectric} \\&{  nano-powder of $Sn_2P_2S_6$''} \\
&  \textsc{Diploma with honors.} {Scientific adviser: Dr. Elena Ouskova}\\
&\\

\end{tabular}

%Section: Scholarships and additional info
\section{Scholarships}
\begin{tabular}{rl}
2011& Student grant of the conference IPAC 2011\\
2009--2012& PhD Student Fellowship of the French Ministry of Education\\
2008--2009& Scholarship of the French Government\\
2002--2008& Scholarship of the National Taras Shevchenko University of Kyiv\\
2007--2008& Scholarship of the foundation Semper Polonia\\ 
\end{tabular}


%\newpage
% --------- Research ----------------------------------------------------

\section{Research interests}
My main research interests lie in the field of accelerator physics and accelerator technology. In particular, the compact X-rays sources based on the Compton scattering are of great interest to me. Modelling and understanding the particle dynamics using different types of scientific approaches and computer simulations. 
Conducting the experiments with the subsequent data analysis.
Working on the control system of the accelerators, mainly the developing the high level applications for machine operation.
Developing and testing the different types of the beam diagnostics. 
%Developing and testing the different types of the beam diagnostics. Writing the high level application for machine operation. Conducting the experiments and data analysis.


% -------- Publication --------------------------------------------

\section{Publications}
\begin{itemize}
%\item[$\triangleright$]
%I. Chaikovska, L. Burmistrov, N. Deleru, A. Variola "\textit{Optical fiber beam loss monitor for the PHIL and ThomX facilities}",  Proceedings of IPAC'14,  Dresden, Germany.
\item[$\triangleright$]
X. Artru, I. Chaikovska, R. Chehab et al.   "\textit{Investigations on a hybrid positron source with a granular converter}",  Nuclear Instruments and Methods B, 355, 60-64, 2015.
\item[$\triangleright$]
I. Chaikovska, A. Variola.  "\textit{Equilibrium energy spread and emittance in a Compton ring: An alternative approach}", Phys. Rev. ST Accel. Beams 17, 044004, 2014.
\item[$\triangleright$]
V. Petrillo, I. Chaikovska, C. Ronsivalle, AR Rossi, L. Serafini, C. Vaccarezza.  "\textit{Phase space distribution of an electron beam emerging from Compton/Thomson back-scattering by an intense laser pulse}", EPL (Europhysics Letters), IOP Publishing, volume 101, P10008, 2013.
%\item[$\triangleright$]
%I. Chaikovska, R. Chehab, O. Dadoun, P. Lepercq, A. Variola, "\textit{Polarized positron source with a Compton multiple interaction point line}", Proceedings of IPAC'12, New Orleans, USA.
\item[$\triangleright$]
V. Petrillo, A. Bacci, R. Ben Al{\`\i} Zinati, I. Chaikovska et al. "\textit{Photon flux and spectrum of $\gamma$-rays  compton sources}", Nuclear Instruments and Methods A, 693, 109-116, 2012.
\item[$\triangleright$]
 I. Chaikovska, T. Akagi, S. Araki, J. Bonis et al. "\textit{Production of gamma rays by pulsed laser beam Compton scattering off GeV-electrons using a non-planar four-mirror optical cavity}",  Journal of Instrumentation, IOP Publishing, volume 7, P01021, 2012.
%\item[$\triangleright$]
%N. Delerue, J. Bonis, I. Chaikovska et al. "\textit{High flux polarized gamma rays production: first measurements with a four-mirror cavity at the ATF}",  Proceedings of IPAC'11,  San Sebastian,~Spain.
%\item[$\triangleright$]
%O. Dadoun, I. Chaikovska, P. Lepercq, F. Poirier,  A. Variola, L.~Rinolfi,  A. Vivoli, V. Strakhovenko, C.~Xu. "\textit{The Baseline Positron Production and Capture Scheme for CLIC}",  Proceedings of IPAC'10, Kyoto, Japan.
% \item[$\triangleright$]  
%A. Variola, C. Bruni, I.Chaikovska, O. Dadoun, R. Chehab, M. Kuriki, T. Omori, J. Urakawa, L.~Rinolfi, A. Vivoli, F. Zimmermann, "\textit{ERL parameters for Compton polarized positron sources}", Proceedings of PAC'09, Vancouver, BC, Canada.
%\item[$\triangleright$] 
%I. Chaikovska, \textit{Optimization of the Compton line for the polarized positron sources}, Proceedings of Trans-European School of High Energy Physics (2009).
\end{itemize}


% ------- Scientific Activity ------------------------------------------------------

\section{Schools and conferences}
\begin{itemize}

\item[$\triangleright$] International Workshop on Polarized Positrons, POSIPOL--2015, 2 - 4 September, Cockcroft Institute, United Kingdom.
\emph{Contribution: ``Investigation on a reliable positron source for ILC and CLIC with the hybrid target ".}

\item[$\triangleright$] International Particle Accelerator Conference, IPAC--2015, 3-- 8 May, Richmond, VA, USA.
\emph{Contribution: ``Lattice correction using LOCO for the ThomX storage ring".}

\item[$\triangleright$]  Compact Linear Collider (CLIC) workshop,  26--30 January 2015, CERN, Geneva, Switzerland. 
\emph{Contribution: ``A hybrid positron source with the granular converter for CLIC ".}

\item[$\triangleright$] International Particle Accelerator Conference, IPAC--2014, 15 -- 20 June, Dresden, Germany.
\emph{Contribution: ``Optical Fiber Beam Loss Monitor for the PHIL and ThomX Facility".}

\item[$\triangleright$] School organised by SLAC, \emph{``Summer seminar on Electron and Photon Beams SSSEPB-2013}'',  Menlo Park, California, USA.

\item[$\triangleright$] International Particle Accelerator Conference, IPAC--2012, 20 -- 25 May, New Orleans, USA.
\emph{Contribution: ``Polarized Positron Source with a Compton Multiple Interaction Point Line".}

\item[$\triangleright$] International Particle Accelerator Conference, IPAC--2011, 4 -- 9 September,  San Sebastian, Spain.
\emph{Contribution: ``Effect of Compton scattering on the electron beam dynamics at the ATF damping ring".}

\item[$\triangleright$] International Workshop on Polarized Positrons, POSIPOL--2011, 28--30 August, IHEP, Beijing, China.
\emph{Contribution: ``High energy gamma production: analysis of LAL four-mirror cavity data".}

\item[$\triangleright$] School organised by CNRS-IN2P3, \emph{\'Ecole IN2P3 d'instrumentation  du d�tecteur \`a  la mesure 2011},  Ol\'eron, France.

\item[$\triangleright$] International Workshop on Linear Colliders, IWLC--2010,  8--22 October, CERN, Geneva \\ Switzerland. 
\emph{``Contribution: `Compact Linear Collider (CLIC) ERL based positron source ".}

\item[$\triangleright$] International Workshop on Polarized Positrons, POSIPOL--2010, May, 31st. -- June, 2nd., KEK, Japan.
\emph{Contribution: ``Positron production and capture with a Compton multiple IPs line".}


\item[$\triangleright$] Joint Universities Accelerator School, \emph{JUAS--2010},  Archamps, France.

\item[$\triangleright$]  Compact Linear Collider (CLIC) workshop,  12--16 October 2009, CERN, Geneva \\ Switzerland. 

%\item[$\triangleright$] Participation in the International scientific conferences and workshops: International Workshop on Linear Colliders, POSIPOL.

\item[$\triangleright$] Trans-European School of High Energy Physics, \emph{TESHEP--2009},  Zakopane, Poland.


 \item[$\triangleright$] Internship at \'Ecole Polytechnique ( \emph{2008}), CMS group of LLR , Palaiseau, France.\\
\emph{``Higgs boson at LHC'' }, working with Prof.~ Ludwik Dobrzynski.

\item[$\triangleright$] Summer school, \emph{Physbio--2007}, \emph{``Non-equilibrium in Physics and in Biology- Systemic approaches to biological physics''},  Saint Etienne de Tine\'{e}, France.

\item[$\triangleright$] \emph{ International Summer Student Program} at GSI  Helmholtz Centre for Heavy Ion Research (\emph{2006}), Darmstadt, Germany.

\end{itemize}

% ------- Skills ------------------------------------------------------

\section{}
\begin{tabular}{l p{12cm}}
 \multicolumn{2}{c}{} \\
 \textsc{Knowledge of:}&Numerical simulation, Monte-Carlo simulations, data acquisition, data analysis.\\
\textsc{}& C++, GEANT4, ROOT, Matlab, Mathematica, \LaTeX.  \\ %Software:
\textsc{}&  CAIN, ASTRA, TRANSPORT, MML, AT.\\ %Accelerator codes:
\textsc{Language:}& English(Fluent spoken/written), French( spoken/written), Polish(spoken), Russian(second language), Ukrainian (mother tongue).\\
\end{tabular}


% ------- References ------------------------------------------------------

\section{References}
\begin{tabular}{p{2.4cm}|p{12cm}}
 \textsc{Alessandro ~Variola} & \emph{ELI-NP-GBS machine leader}, Istituto Nazionale di Fisica Nucleare, Laboratori Nazionali di Frascati, Italy, \href{mailto:variola@lal.in2p3.fr}{alessandro.variola@lnf.infn.it}\\
 
  \multicolumn{2}{c}{} \\
  \textsc{Achille ~Stocchi} & \emph{Director of the Laboratoire de l'Acc\'el\'erateur Lin\'eaire d'Orsay}, IN2P3/CNRS, Universit\'e Paris-Sud XI, \href{mailto:variola@lal.in2p3.fr}{stocchi@lal.in2p3.fr}\\
 
% \multicolumn{2}{c}{} \\
% \textsc{Nicolas ~Delerue} & \emph{Ph.D in Physics}, IN2P3/CNRS,  Laboratoire de l'Acc\'el\'erateur Lin\'eaire, \href{mailto:delerue@lal.in2p3.fr}{delerue@lal.in2p3.fr}\\
\end{tabular}
%\section{Interests and Activities}
%Technology, Open-Source, Programming\\
%Paradoxes in Decision Making, Psychoanalysis, Behavioural Finance\\
%Football, Travelling

\end{document}

[ForwardSearch("%bm.pdf","%Wc",%l,0)]
